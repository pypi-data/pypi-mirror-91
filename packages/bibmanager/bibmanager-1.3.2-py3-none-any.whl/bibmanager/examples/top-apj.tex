% latbibdo  template

\newcommand\apjcls{1}
\newcommand\aastexcls{2}
\newcommand\othercls{3}

% Select ony one pair of \papercls and class file:

% AASTEX61 cls:
\newcommand\papercls{\aastexcls}
%\documentclass[tighten, times, twocolumn]{aastex62}
\documentclass[tighten, times, trackchanges, twocolumn]{aastex62}
%\documentclass[tighten, times, twocolumn, twocolappendix]{aastex62}
%\documentclass[tighten, times, manuscript]{aastex62} % Onecolumn, doublespaced

% Emulate ApJ cls:
%\newcommand\papercls{\apjcls}
%\documentclass[iop]{emulateapj}

% Other cls:
%\newcommand\papercls{\othercls}
%\documentclass[letterpaper,12pt]{article}


%% :::::::::::::::::::::::::::::::::::::::::::::::::::::::::::::::::::
% These are latex packages that enable various capability.

\if\papercls \apjcls
\usepackage{apjfonts}
\else\if\papercls \othercls
\usepackage{epsfig}
\usepackage{margin}
% times font (for \othercls):
\usepackage{times}
\fi\fi
\usepackage{ifthen}
\usepackage{natbib}
\usepackage{amssymb}
\usepackage[fleqn]{amsmath}
\usepackage{appendix}
\usepackage{etoolbox}
\usepackage[T1]{fontenc}
\usepackage{paralist}

% This one defines a few more journals (DPS and AAS abstracts) for bibtex:
\if\papercls \apjcls
\newcommand\aas{\ref@jnl{AAS Meeting Abstracts}}% *** added by jh
          % American Astronomical Society Meeting Abstracts
\newcommand\dps{\ref@jnl{AAS/DPS Meeting Abstracts}}% *** added by jh
          % American Astronomical Society/Division for Planetary Sciences Meeting Abstracts
\newcommand\maps{\ref@jnl{MAPS}}% *** added by jh
          % Meteoritics and Planetary Science
\else\if\papercls \othercls
\usepackage{astjnlabbrev-jh}
\fi\fi

% Bibliographystyle chooses a bibtex .bst file, which defines the
% format of the references.  It's important to pick one that works for
% the journal you are writing for and that has hyperlinks for the
% actual paper online.
\bibliographystyle{apj_hyperref}
%\bibliographystyle{aasjournal}

% Enable this for packed reference list:
%\setlength\bibsep{0.0pt}

% Enable this to remove section numbers:
%\setcounter{secnumdepth}{0}

%% % Enable this for bullet-point separated references:
%% \usepackage{paralist}
%% \renewenvironment{thebibliography}[1]{\let\par\relax%
%%   \section*{\refname}\inparaitem}{\endinparaitem}
%% \let\oldbibitem\bibitem
%% \renewcommand{\bibitem}{\item[\textbullet]\oldbibitem}


% Setup hyperreferences style:
\if\papercls \aastexcls
\hypersetup{citecolor=blue, % color for \cite{...} links
            linkcolor=blue, % color for \ref{...} links
            menucolor=blue, % color for Acrobat menu buttons
            urlcolor=blue}  % color for \url{...} links
\else
\usepackage[%pdftex,      % hyper-references for pdflatex
bookmarks=true,           %%% generate bookmarks ...
bookmarksnumbered=true,   %%% ... with numbers
colorlinks=true,          % links are colored
citecolor=blue,           % color for \cite{...} links
linkcolor=blue,           % color for \ref{...} links
menucolor=blue,           % color for Acrobat menu buttons
urlcolor=blue,            % color for \url{...} links
linkbordercolor={0 0 1},  %%% blue frames around links
pdfborder={0 0 1},
frenchlinks=true]{hyperref}
\fi

% These macross generate the hyperlinks in the References section:
\if\papercls \othercls
\newcommand{\eprint}[1]{\href{http://arxiv.org/abs/#1}{#1}}
\else
\renewcommand{\eprint}[1]{\href{http://arxiv.org/abs/#1}{#1}}
\fi
\newcommand{\ISBN}[1]{\href{http://cosmologist.info/ISBN/#1}{ISBN: #1}}
\providecommand{\adsurl}[1]{\href{#1}{ADS}}

% hyper ref only the year in citations:
\makeatletter
% Patch case where name and year are separated by aysep
\patchcmd{\NAT@citex}
  {\@citea\NAT@hyper@{%
     \NAT@nmfmt{\NAT@nm}%
     \hyper@natlinkbreak{\NAT@aysep\NAT@spacechar}{\@citeb\@extra@b@citeb}%
     \NAT@date}}
  {\@citea\NAT@nmfmt{\NAT@nm}%
   \NAT@aysep\NAT@spacechar\NAT@hyper@{\NAT@date}}{}{}

% Patch case where name and year are separated by opening bracket
\patchcmd{\NAT@citex}
  {\@citea\NAT@hyper@{%
     \NAT@nmfmt{\NAT@nm}%
     \hyper@natlinkbreak{\NAT@spacechar\NAT@@open\if*#1*\else#1\NAT@spacechar\fi}%
       {\@citeb\@extra@b@citeb}%
     \NAT@date}}
  {\@citea\NAT@nmfmt{\NAT@nm}%
   \NAT@spacechar\NAT@@open\if*#1*\else#1\NAT@spacechar\fi\NAT@hyper@{\NAT@date}}
  {}{}
\makeatother

% Define lowcase: a MakeLowercase that doesn't break on subtitles:
\makeatletter
\DeclareRobustCommand{\lowcase}[1]{\@lowcase#1\@nil}
\def\@lowcase#1\@nil{\if\relax#1\relax\else\MakeLowercase{#1}\fi}
\pdfstringdefDisableCommands{\let\lowcase\@firstofone}
\makeatother

% unslanted mu, for ``micro'' abbrev.
\DeclareSymbolFont{UPM}{U}{eur}{m}{n}
\DeclareMathSymbol{\umu}{0}{UPM}{"16}
\let\oldumu=\umu
\renewcommand\umu{\ifmmode\oldumu\else\math{\oldumu}\fi}
\newcommand\micro{\umu}
\if\papercls \othercls
\newcommand\micron{\micro m}
\else
\renewcommand\micron{\micro m}
\fi
\newcommand\microns{\micron}
\newcommand\microbar{\micro bar}

% These define commands outside of math mode:
% \sim
\let\oldsim=\sim
\renewcommand\sim{\ifmmode\oldsim\else\math{\oldsim}\fi}
% \pm
\let\oldpm=\pm
\renewcommand\pm{\ifmmode\oldpm\else\math{\oldpm}\fi}
% \times
\newcommand\by{\ifmmode\times\else\math{\times}\fi}
% Ten-to-the-X and times-ten-to-the-X:
\newcommand\ttt[1]{10\sp{#1}}
\newcommand\tttt[1]{\by\ttt{#1}}

% A tablebox lets you define some lines in a block, using \\ to end
% them.  The block moves as a unit.  Good for addresses, quick lists, etc.
\newcommand\tablebox[1]{\begin{tabular}[t]{@{}l@{}}#1\end{tabular}}
% These commands are blank space exactly the width of various
% numerical components, for spacing out tables.
\newbox{\wdbox}
\renewcommand\c{\setbox\wdbox=\hbox{,}\hspace{\wd\wdbox}}
\renewcommand\i{\setbox\wdbox=\hbox{i}\hspace{\wd\wdbox}}
\newcommand\n{\hspace{0.5em}}

% \marnote puts a note in the margin:
\newcommand\marnote[1]{\marginpar{\raggedright\tiny\ttfamily\baselineskip=9pt #1}}
% \herenote makes a bold note and screams at you when you compile the
% document.  Good for reminding yourself to do something before the
% document is done.
\newcommand\herenote[1]{{\bfseries #1}\typeout{======================> note on page \arabic{page} <====================}}
% These are common herenotes:
\newcommand\fillin{\herenote{fill in}}
\newcommand\fillref{\herenote{ref}}
\newcommand\findme[1]{\herenote{FINDME #1}}

% \now is the current time.  Convenient for saying when the draft was
% last modified.
\newcount\timect
\newcount\hourct
\newcount\minct
\newcommand\now{\timect=\time \divide\timect by 60
         \hourct=\timect \multiply\hourct by 60
         \minct=\time \advance\minct by -\hourct
         \number\timect:\ifnum \minct < 10 0\fi\number\minct}

% This is pretty much like \citealp
\newcommand\citeauthyear[1]{\citeauthor{#1} \citeyear{#1}}

% These are short for multicolumn, to shorten the length of table lines.
\newcommand\mc{\multicolumn}
\newcommand\mctc{\multicolumn{2}{c}}

% Joetex character unreservations.
% This file frees most of TeX's reserved characters, and provides
% several alternatives for their functions.

% Tue Mar 29 22:23:03 EST 1994
% modified 12 Oct 2000 for AASTeX header

% utility
\catcode`@=11

% Define comment command:
\newcommand\comment[1]{}

% Undefine '%' as special character:
\newcommand\commenton{\catcode`\%=14}
\newcommand\commentoff{\catcode`\%=12}

% Undefine '$' as special character:
\renewcommand\math[1]{$#1$}
\newcommand\mathshifton{\catcode`\$=3}
\newcommand\mathshiftoff{\catcode`\$=12}

% Undefine '&' as special character:
\let\atab=&
\newcommand\atabon{\catcode`\&=4}
\newcommand\ataboff{\catcode`\&=12}

% Define \sp and \sb for superscripts and subscripts:
\let\oldmsp=\sp
\let\oldmsb=\sb
\def\sp#1{\ifmmode
           \oldmsp{#1}%
         \else\strut\raise.85ex\hbox{\scriptsize #1}\fi}
\def\sb#1{\ifmmode
           \oldmsb{#1}%
         \else\strut\raise-.54ex\hbox{\scriptsize #1}\fi}
\newbox\@sp
\newbox\@sb
\def\sbp#1#2{\ifmmode%
           \oldmsb{#1}\oldmsp{#2}%
         \else
           \setbox\@sb=\hbox{\sb{#1}}%
           \setbox\@sp=\hbox{\sp{#2}}%
           \rlap{\copy\@sb}\copy\@sp
           \ifdim \wd\@sb >\wd\@sp
             \hskip -\wd\@sp \hskip \wd\@sb
           \fi
        \fi}
\def\msp#1{\ifmmode
           \oldmsp{#1}
         \else \math{\oldmsp{#1}}\fi}
\def\msb#1{\ifmmode
           \oldmsb{#1}
         \else \math{\oldmsb{#1}}\fi}

% Undefine '^' as special character:
\def\supon{\catcode`\^=7}
\def\supoff{\catcode`\^=12}

% Undefine '_' as special character:
\def\subon{\catcode`\_=8}
\def\suboff{\catcode`\_=12}

\def\supsubon{\supon \subon}
\def\supsuboff{\supoff \suboff}

% Undefine '~' as special character:
\newcommand\actcharon{\catcode`\~=13}
\newcommand\actcharoff{\catcode`\~=12}

% Undefine '#' as special character:
\newcommand\paramon{\catcode`\#=6}
\newcommand\paramoff{\catcode`\#=12}

\comment{And now to turn us totally on and off...}

\newcommand\reservedcharson{ \commenton  \mathshifton  \atabon  \supsubon 
                             \actcharon  \paramon}

\newcommand\reservedcharsoff{\commentoff \mathshiftoff \ataboff \supsuboff 
                             \actcharoff \paramoff}

\newcommand\nojoe[1]{\reservedcharson #1 \reservedcharsoff}
\catcode`@=12
\reservedcharson


\if\papercls \apjcls
\newcommand\widedeltab{deluxetable}
\else
\newcommand\widedeltab{deluxetable*}
\fi


%% :::::::::::::::::::::::::::::::::::::::::::::::::::::::::::::::::::
%% Convenience macross:
\newcommand\tnm[1]{\tablenotemark{#1}}
%% Spitzer:
\newcommand\SST{{\em Spitzer Space Telescope}}
\newcommand\Spitzer{{\em Spitzer}}
%% chi-squared:
\newcommand\chisq{\ifmmode{\chi\sp{2}}\else\math{\chi\sp{2}}\fi}
\newcommand\redchisq{\ifmmode{ \chi\sp{2}\sb{\rm red}}
                    \else\math{\chi\sp{2}\sb{\rm red}}\fi}
%% Equilibrium temperature:
\newcommand\Teq{\ifmmode{T\sb{\rm eq}}\else$T$\sb{eq}\fi}
%% Jupiter mass, radius:
\newcommand\mjup{\ifmmode{M\sb{\rm Jup}}\else$M$\sb{Jup}\fi}
\newcommand\rjup{\ifmmode{R\sb{\rm Jup}}\else$R$\sb{Jup}\fi}
%% Solar mass, radius:
\newcommand\msun{\ifmmode{M\sb{\odot}}\else$M\sb{\odot}$\fi}
\newcommand\rsun{\ifmmode{R\sb{\odot}}\else$R\sb{\odot}$\fi}
%% Earth mass, radius:
\newcommand\mearth{\ifmmode{M\sb{\oplus}}\else$M\sb{\oplus}$\fi}
\newcommand\rearth{\ifmmode{R\sb{\oplus}}\else$R\sb{\oplus}$\fi}
%% Molecules:
\newcommand\molhyd{\ifmmode{{\rm H}\sb{2}}\else{H$\sb{2}$}\fi}
\newcommand\methane{\ifmmode{{\rm CH}\sb{4}}\else{CH$\sb{4}$}\fi}
\newcommand\water{\ifmmode{{\rm H}\sb{2}{\rm O}}\else{H$\sb{2}$O}\fi}
\newcommand\carbdiox{\ifmmode{{\rm CO}\sb{2}}\else{CO$\sb{2}$}\fi}
\newcommand\carbmono{\ifmmode{{\rm CO}}\else{CO}\fi}
\newcommand\ammonia{\ifmmode{{\rm NH}\sb{3}}\else{NH$\sb{3}$}\fi}
\newcommand\acetylene{\ifmmode{{\rm C}\sb{2}{\rm H}\sb{2}}
                        \else{C$\sb{2}$H$\sb{2}$}\fi}
\newcommand\ethylene{\ifmmode{{\rm C}\sb{2}{\rm H}\sb{4}}
                        \else{C$\sb{2}$H$\sb{4}$}\fi}
\newcommand\cyanide{\ifmmode{{\rm HCN}}\else{HCN}\fi}
\newcommand\nitrogen{\ifmmode{{\rm N}\sb{2}}\else{N$\sb{2}$}\fi}


%% Units:
\newcommand\ms{m\;s$\sp{-1}$}
\newcommand\cms{cm\;s$\sp{-1}$}

\newcommand\degree{\degr}
\newcommand\degrees{\degree}
\newcommand\vs{\emph{vs.}}

